\section{Quiz 3}
\begin{question}[type=exam]
What is the primary purpose of free software license?
\begin{itemize}
\chk To allow free distribution of modifications made to the free software. Therefore to contribute to the community's shared knowledge.
\chk To give the user back their freedom taken by copyright law.
\chk To free the developers of the software from any obligations and claims which can be risen by their users.
\end{itemize}
\end{question}
\begin{solution}
To give the user back their freedom taken by copyright law.
\end{solution}


\begin{question}[type=exam]
Which problem is addressed by Affero GPL which is not in scope of GPL?
\begin{itemize}
\chk If the modified software is executed in the server and is used by others, modifications of the server software shall also be available to those "others".
\chk Software libraries distributed under GPL force the packages which use them to be licensed under GPL as well. AGPL addresses this issue by making copyleft conditions weaker than in original GPL.
\chk GNU system needed non copyleft licenses to accompany copyleft (which is GPL) because some of the GNU projects wanted to use non copyleft licenses. AGPL is an answer to this demand.
\end{itemize}
\end{question}
\begin{solution}
If the modified software is executed in the server and is used by others, modifications of the server software shall also be available to those "others".
\end{solution}


\begin{question}[type=exam]
Is Mozilla Public License (MPL) a copyleft and why? Select one answer.
\begin{itemize}
\chk Yes. It makes it impossible to make free program non free.
\chk No. It is possible to make free program non free if original program is released under MPL.
\chk Yes. Because it requires the original source code to be distributed under MPL license as well.
\end{itemize}
\end{question}
\begin{solution}
No. It is possible to make free program non free if original program is released under MPL.
\end{solution}


\begin{question}[type=exam]
Which non copyleft license is recommended by FSF to be used with projects which needs to be distributed under non copyleft freesoftware license?
\begin{itemize}
\chk X11 License.
\chk Apache 2.0 License.
\chk GNU AGPL.
\end{itemize}
\end{question}
\begin{solution}
Apache 2.0 License.
\end{solution}


\begin{question}[type=exam]
What was the reason for distribution Ogg/Vorbis codec under non copyleft license?
\begin{itemize}
\chk It was very important to extend the range of music players supporting Ogg/Vorbis format.
\chk It was decided to test and measure the effect of selecting non copyleft license on software distribution coverage.
\chk It was a huge mistake --- developers of the codec were not aware of the disadvantages of non copyleft licenses as explained in our lectures.
\end{itemize}
\end{question}
\begin{solution}
It was very important to extend the range of music players supporting Ogg/Vorbis format.
\end{solution}


\begin{question}[type=exam]
What is called "Secondary Effect of Copyleft"?
\begin{itemize}
\chk Copyleft gives users the Second Freedom --- the freedom to study how the program works, and change it so it does their computing as they wish.
\chk Copyleft prohibits limiting users freedom and as secondary effect proprietary software developers don't like to use and modify the software which is distributed under copyleft.
\chk Modifications to the program distributed under copyleft license can be included into the original program. Therefore copyleft facilitates contribution to the communitys shared knowledge.
\end{itemize}
\end{question}
\begin{solution}
Modifications to the program distributed under copyleft license can be included into the original program. Therefore copyleft facilitates contribution to the communitys shared knowledge.
\end{solution}


\begin{question}[type=exam]
What is called "pushover license" and why?
\begin{itemize}
\chk These free software licenses are very weak non copyleft licenses. They are called this way because they permit almost everything.
\chk This is another name for copyleft licenses. It emphasizes the fact that these licenses are pushing the freedom to the users.
\chk This name is used for proprietary licenses. It emphasizes the fact that proprietary software is pushed to the users with the help of these licenses.
\end{itemize}
\end{question}
\begin{solution}
These free software licenses are very weak non copyleft licenses. They are called this way because they permit almost everything.
\end{solution}


\begin{question}[type=exam]
I want to become SaaSS user. Server code is distributed under AGPL. Am I safe?
\begin{itemize}
\chk Yes. Because server code is distributed under free software license and I have access to the server software source code.
\chk No. No license can make SaaSS ethical. My computing is still done by the server and my privacy is under threat, for example.
\chk No. AGPL is not a license which can be used to protect the users from SaaSS. Other license (which is mentioned in the additional materials for this course) shall be used to achieve this effect.
\end{itemize}
\end{question}
\begin{solution}
No. No license can make SaaSS ethical. My computing is still done by the server and my privacy is under threat, for example.
\end{solution}


\begin{question}[type=exam]
Why it's a bad idea to state in the source code "This source code is distributed under BSD license"? Select one answer.
\begin{itemize}
\chk There is more than one BSD license. It's better to be specific.
\chk It's not enough to make such a one-line statement. I need to include the text of the license in every source code file.
\chk Because in this case all the users of this software will need to mention "Berkley University" in all the advertisings of their products (based on this software).
\end{itemize}
\end{question}
\begin{solution}
There is more than one BSD license. It's better to be specific.
\end{solution}


\begin{question}[type=exam]
What happens if I publish my software without any license?
\begin{itemize}
\chk My software will give freedom to the users. They will exercise four essential freedoms described in the lecture 1.2.
\chk My software will be in so called "public domain". Hence, it will give freedom to the users.
\chk My software will be automatically copyrighted. Others will not be able to exercise all the four essential freedoms described in the lecture 1.2.
\end{itemize}
\end{question}
\begin{solution}
My software will be automatically copyrighted. Others will not be able to exercise all the four essential freedoms described in the lecture 1.2.
\end{solution}



